% Created 2022-11-04 Fri 13:21
% Intended LaTeX compiler: pdflatex
\documentclass[11pt]{article}
\usepackage[utf8]{inputenc}
\usepackage[T1]{fontenc}
\usepackage{graphicx}
\usepackage{longtable}
\usepackage{wrapfig}
\usepackage{rotating}
\usepackage[normalem]{ulem}
\usepackage{amsmath}
\usepackage{amssymb}
\usepackage{capt-of}
\usepackage{hyperref}
\author{Scritte da Qualcuno}
\date{L'altro giorno}
\title{Cose}
\hypersetup{
 pdfauthor={Scritte da Qualcuno},
 pdftitle={Cose},
 pdfkeywords={},
 pdfsubject={},
 pdfcreator={Emacs 28.1 (Org mode 9.5.5)}, 
 pdflang={English}}
\begin{document}

\maketitle
\tableofcontents


\section{Viste Aggiornabili}
\label{sec:org43cf4d2}
cose scritte col create view per creare una vista che modifica lo schema esterno della
base dati, dal punto di vista delle query questa vista è a tutti gli effetti una tabella
della base, risulta in qualche modo la domanda, posso usare questa vista per modificare
l'istanza della base dati.

posso fare set, oltre a get, di elementi della tabella?
dal punto di vista formale, una vista è una funzione che estrae roba da una tabella

\(vista = F(tabella)\)

se posso determinare in maniera univoca qual'è l'aggiornamento della base dati che implica
quell'aggiornamento della vista, allora posso aggiornare la tabella, allora la vista può
essere aggiornabile, perchè invertibile, perchè posso arrivare univocamente alla riga che
mi ha dato quella cosa nella tabella

condizioni, vincoli, che valendo questi vincoli la vista è aggiornabile, non si pretende
di individuare tutti i casi in cui è invertibile, ma si definiscono delle condizioni
sufficienti per dire "ok la vista è \emph{x}, quindi la vista è aggiornabile"

questi sono i limiti posti da dbms per considerare una vista come aggiornabile, quando le
considerano aggiornabili
\begin{itemize}
\item assenza di group by
\item assenza di funzioni aggregate
\item assenza di distinct
\item assenza di join, impliciti o espliciti
\end{itemize}

quello a cui questi join si riferiscono è della parte esterna della vista, quindi
\begin{itemize}
\item solo una tabella
\item niente distinct
\item niente aggregati
\item niente raggruppamento
\end{itemize}

\subsection{Esempio}
\label{sec:org1e7e039}
supponiamo di avere una base dati con tabelle   
\begin{itemize}
\item imp : codice, nome, sede, ruolo, stipendio
\item sedi : sede, responsabile, città
\end{itemize}

la città dove vanno a lavorare gli impiegati dipende da dove è localizzata la sede
c'è un vincolo di chiave

definiamo una vista per tutti gli impiegati che lavorano a bologna
\begin{verbatim}
create view impBo(...)
select I.*
from impiegato I, join sedi S on (I.sede = S.sede)
where S.città = 'Bologna'
\end{verbatim}

questa vista non è aggiornabile, nella clausola \texttt{from} hai due tabelle, quindi c'è un join
implicito, questa vista non è aggiornabile per sti criteri.

ma formuliamola con una query annidata,
\begin{verbatim}
where impiegato.sede in (select le sedi a bologna)
\end{verbatim}
oppure
\begin{verbatim}
create view ImpBo(...)
select I.*
from impiegato I where I.sede in (select S.sede from Sedi where S.città = 'Bologna')
\end{verbatim}
in questo caso possiamo vedere che la query annidante rispetta tutti i vincoli di
aggiornaiblità, quindi sta query che da le stesse informazioni ora è aggiornabile

nel caso tu debba usare più tabelle, se la vuoi aggiornabile, devi vedere se puoi farla
con select annidate

\subsection{Aggiornare query}
\label{sec:org6b947d0}
\begin{verbatim}
insert into ImpBo(...) values(<dati di impiegato>)
\end{verbatim}

ok, ma questa è una vista virtuale, non esiste come tabella, quindi la tupla specificata
dove mi va a finire? la vista non ha un istanza materializzata, la vista è un getter che
si pensa figo, per avere che quel getter dia quella tupla in più la tupla deve andare a
finire nella tabella \texttt{imp}, e abbiamo potuto determinare dove doveva andare la tupla
apputno perchè la vista era definita da una query reversibile.

se io dopo questo inserimento in \texttt{imp} facessi \texttt{select * from ImpBo}, allora vedrei che
l'impiegato è stato aggiunto.

facciamo un'altra query del tipo
\begin{verbatim}
insert into ImpBo(...) values(<dati di impiegato>)
\end{verbatim}

ma mettiamo che stavolta il codice della sede dato sia un codice di sede che già esiste,
ma non si trova a bologna, aggiungo agli impiegati di bologna un impiegato che lavora
nella sede di milano.
Questa cosa possiamo vedere che non funziona benissimo
l'inserimento non rispetta la query che definisce la vista

il controsenso viene dal fatto che la tupla non rispetta la select di definizione della
vista, quando recupero i contenuti della vista quei dati non ci sono, non rispettano il
comportamento della vista.

quindi nella definizione di una vista aggiornabile si può mettere un check per inserimento
e aggiornamento, \texttt{with check option}, devo vedere se quell'aggiornamento è compatibile con
la definizione della vista, così quando cerco di inserire o modificare una vista.

Se il dato inserito nella vista sarà visibile attraverso la definizione (\texttt{select}) della
vista allora va bene, se il dato inserito non sarà visibile attraverso la definizione
(\texttt{select}) della vista allora errore, qui il tizio di milano non sarà visibile dalla
tabella di "vedi quelli di bologna", quindi errore.

visto che una vista puoi definirla in vista di altre viste, puoi avere due tipi di \texttt{with
check option} 
\begin{itemize}
\item \texttt{with local check option}, in vista di questa vista
\item \texttt{with cascaded check option}, in vista di questa vista e delle altre viste da cui è
definita.
\end{itemize}

ci sono qundi dei criteri in base ai quali la vista è aggiornabile.

Passo primo da ricordare, l'eventuale riga che viene aggiunta viene in realtà memorizzata
nella tabella di base sulla quale la vista è definita, non nella vista, che è una vista
virtuale.
Nel caso delle viste aggiornabili è poi importante vedere se l'inserimento è compatibile o
meno con la select di definzione della vista, se voglio che ci sia consistenza, e che
quindi i dati inseriti nella vista siano visibili dalla vista, devo definire la vista con
\texttt{with check option}, che obbliga il dbms a controllare che l'aggiornamento funzoini.

se non metto la with check option allora o
\begin{itemize}
\item la vista è read only
\item la vista farà un po' i cazzi suoi
\end{itemize}

\section{Ora fa prove compitini}
\label{sec:org0226520}
puoi partecipare al compitino anche se non hai superato algoritmi, ma il compitino vale
solo fino a gennaio febbraio, quindi se non passi algoritmi per sta sessione il compitino
bellissimo, non ci fai niente.

se non hai dato algoritmi non puoi partecipare al preappello di dicembre, ad esempio

\subsection{Prova compitino VII}
\label{sec:org15df7e6}
\subsubsection{Schema}
\label{sec:org3cdd896}
\begin{itemize}
\item Policeman : \uline{code}, name, surname, address, areaCode
\item Fine : \uline{code}, policeman, car, date, infraction, streetName, areaCode, cost
\item Infraction : \uline{code}, description
\item Car : \uline{plateNumber}, type, owner
\item Owner : \uline{code}, name, surname, address, areaCode
\item AreaCode : \uline{code}, cityName, provinceName
\end{itemize}

(nelle query usa CAP e areaCode come sinonimi)
\subsubsection{Prima query}
\label{sec:orgcb41b94}
per ogni cap, il codice ed il numero di multe medie fatte al giorno
(ammissione dal Pala, la definizione \textpm{} torna \textpm{} non torna tantissimo)

da calcolare un numero, e la media di quel numero, mi sono un po' perso, so che va fatta
\begin{enumerate}
\item Pala dixit
\label{sec:orgedacd9e}

crea una vista in cui per ogni cap, e per ogni giorno, si contano le multe fatte
\begin{verbatim}
create view multeAreaData(areaCode, date, numero) as
select areaCode, date, count(code)
from fine
group by areaCode, date

select areaCode, avg(numero)
from multeAreaData
group by areaCode
\end{verbatim}

qui visto che code è chiave primaria non cambia fare
\begin{itemize}
\item \texttt{count}
\item \texttt{count *}
\item \texttt{count distinct}
\end{itemize}

se vuoi calcolare anche le aree per cui non è mai stato fatta una multa fai left join di
qualcosa, poi count viene 0 nella vista, e avg verrà 0 nella select.
Mi sono perso i dettagli, non stavo seguendo molto, pardon.
\end{enumerate}

\subsubsection{Seconda query}
\label{sec:org64a0180}
Il cap dove sono state fatte il maggior numero di multe, insieme a quel numero di multe

qui farei una query per legare cap e multe
\begin{verbatim}
create view capMulte(cap, multe) as
select areaCode, count(code)
from fine
where areaCode is not null -- se vuoi evitare problemi quando non specificano areaCode 
group by areaCode

select cap from capMulte where multe = (select max multe from capMulte)
\end{verbatim}
\begin{enumerate}
\item Pala dixit
\label{sec:org87369cf}
una possibile soluzoine è calcolare il numero di multe fatte per cap, e poi vedere il cap
per cui è massimale
\begin{verbatim}
create view multeArea(areaCode, numero) as
select areaCode, count(code)
-- count(code) permetterà di dare 0 per gli areaCode che non hanno mai avuto multe
from fine
group by areaCode

select areCode, numero
from multeArea
where numero = (select max(numero) from multeArea)
\end{verbatim}

altrimenti

\begin{verbatim}
having count(multe) >= all(select count(multe)
\end{verbatim}

\begin{verbatim}
...
(select max(pippo) -- rinominato in un momento di euforia (a detta del Pala in persona)
from (select count(code) as pippo
from fine
group by areaCode))
\end{verbatim}
\end{enumerate}

\subsubsection{Terza query}
\label{sec:org902778a}
per ogni tipo di infrazione, il codice ed il numero medio di multe fatte in un giorno,
\texttt{fine.infraction} indica il tipo di multa.

(come prima, il numero medio di multe fatte in un giorno, assumendo che sia stata fatta
una multa per ogni tipo di infrazione)

come al solito si divide in vista che definisce il coso di cui fare la media
\begin{verbatim}
create view contatore(inf, dat, numero) as
select infraction, date count(fine.code)
from infraction
group by code, date
\end{verbatim}
e fare la media dalla vista poi si fa la media
\begin{verbatim}
select inf, avg(numero)
from contatore
group by inf
\end{verbatim}
\begin{enumerate}
\item Pala dixit
\label{sec:org5519fc1}

vanno contatae usando una vista, il numero per ogni giorno e per ogni infrazione, poi si
fa la media al variare di cosa
si raggruppa per infraction, al variare di date, quindi nella seconda query togli il group
by date

\begin{verbatim}
create view multeTipoData(infraction, data, numero) as
select infraction, date, count(code)
from fine
group by infraction, date

select infraction, avg(numero)
from multeTipData
group by infraction
\end{verbatim}
\end{enumerate}

\subsubsection{Quarta query}
\label{sec:org0b0ad21}
codice e descrizione del tipo di infrazione per cui sono state emesse il maggior numero di
multe, inseme a tale numero

creiamo una vista del codice massimo e numero
\begin{verbatim}
create view infMaxNum(inf, num) as
select infraction, count(code)
from fine
where count(code) >= all(select count(code) from fine group by infraction)
group by infraction
\end{verbatim}

poi abbiamo il codice e bisogna fare il join con la tabella infraction per avere anche la
descrizione, quindi la select vera e propia.
\begin{verbatim}
select inf,descrizione,num
from infMaxNum, infraction
where infMaxNum.inf = infraction.code
\end{verbatim}

\begin{enumerate}
\item Pala dixit
\label{sec:org4f2ce6f}
\begin{verbatim}
-- non so perchè ma gli andava di usare la tabella definita prima
-- la cosa non è necessaria, te fai come vuoi

-- questa basta definirla come "per ogni infrazione il numero di multe fatte"
create view multeTipo(infraction, numero) as
select infraction, sum(numero)
from multeTipoData
group by infraction


select infraction, numero
from multeTipo
where numero = (select max(numero) from multeTipo)
-- oppure
where numero >= all(select numero from multeTipo)
-- molti gradi di libertà
\end{verbatim}


\begin{verbatim}
create view multeTipo (infraction, description, numero) as
select i.code, i.description, count(f.code)
from  infraction left join fine f on i.code = f.infraction
group by i.code, i.description
\end{verbatim}
ai fini di questa select il join normale andava bene
se dovevo contare quando certi tipi di multe non hanno multe fatte, allora avrei fatto un
join esterno per considerare anche quei tipi

di solito posso usare indifferentemente i cosi della condizione, quelle righe avranno i
campi di infraction a valori non nulli mentre i corrispondenti campi di fine saranno
nulli, quindi qeulla condizione non sarà verificata per le righe obbligate e non ho capito
che diceva

poi la query principale sarà
\begin{verbatim}
select infraction, description, numero
from multeTipo
where numero = (select max(numero) from multeTipo)
\end{verbatim}
\end{enumerate}

\subsubsection{Quinta query}
\label{sec:org75c28f4}
codice, nome, e cognome di ogni vigile, e costo totale delle multe fatte, ordinando i
valori descrescenti del costo

io avrei fatto questa
\begin{verbatim}
select p.code, p.name, p.surname, sum(f.cost) costo
from policeman p, fine f
where f.policeman = p.code
group by p.code, p.name, p.surname
order by sum(f.cost) desc
\end{verbatim}

\begin{enumerate}
\item Pala dixit
\label{sec:org0eccd79}
serve la tabella policeman per i dati, e fine per il costo

\begin{verbatim}
select p.code, p.name, p.surname, sum(f.cost) totale
from policeman p, fine f
group by p.code, p.name, p.surname
group by p.code, p.name, p.surname
order by totale desc -- desc necessario perchè di default il criterio è asc, ascendente
\end{verbatim}

errata da parte di Sergio Cibecchini del join esterno
\begin{verbatim}
... -- stesso di sopra
from policeman p left join fine f on f.policeman = p.code
... --stesso di spora

\end{verbatim}
\end{enumerate}

\subsection{Prova compitino IX}
\label{sec:org065b36d}
era un compito a due file, quindi le query sono identiche a due a due
\subsubsection{Schema}
\label{sec:org084f9a2}
(la descrizione oltre alle tabelle, di che significa ogni tabella, è un saggio, vatti a
vedere te le slide, tanto questo è il pdf delle prove compiti che ha messo su moodle)
\begin{itemize}
\item \textbf{Paziente} : \uline{cf}, nome, cognome, indirizzo, città, telefono
\item \textbf{Medico} : \uline{cf}, nome, cognome, indirizzo, città, telefono
\item \textbf{PrenotazoineVisita} : \uline{data}, \uline{ora},  paziente, medico
\item \textbf{Visita} : \uline{codice}, data, paziente, medico, costo
\item \textbf{DettaglioVisita} : \uline{visita}, \uline{controllo},  inNorma, descrizione
\item \textbf{Prescrizione} : \uline{visita}, \uline{farmaco},  posologia
\item \textbf{Controllo} : \uline{codice}, nome
\item \textbf{Farmaco} : \uline{codice}, nome
\end{itemize}

\subsubsection{Prima query}
\label{sec:org554c2c4}
il numero delle visite al termine delle quali non è stato prescritto il farmaco F018

facciamo una vista di tutte le visite in cui\ldots{}

\begin{verbatim}
create view visiteInCuiNon(visita)
select visita.codice
except
(select prescrizione.visita
where farmaco = F018)

select count(visita)
from visitaInCuiNon
\end{verbatim}

se vuoi fare lo scemo fai

\begin{verbatim}
select (select count (*) from visita) -
(select count(*) from prescrizione where farmaco = F018)
as questaCosaFunziona -- ci ho provato nel database impiegati e a postgres va bene
\end{verbatim}

poi la solita epopea del tipo di count non l'ho considerata, ma visto che \uline{visita},
\uline{codice} è una superchiave non serve il \texttt{count distinct}, per ogni visita ogni codice
compare al più una volta.

Questa minchiata è stata approvata dal Pala.

\begin{enumerate}
\item Pala dixit
\label{sec:org670a259}
qui non è decisivo che sia prescritta qualcosa che non sia F018, visto che posso
prescrivere qualcosa \texttt{<> F018} insieme a qualcosa che è \texttt{= F018}
\begin{verbatim}
select count(codice)
from visita
where codice not in(
select visita
from prescrizione
where farmaco = 'F018')
\end{verbatim}
\end{enumerate}

\subsubsection{Seconda query}
\label{sec:orge80ba2b}
Il numero di visite in cui non è stato fatto il controllo C014

è identica alla query di sopra

\begin{enumerate}
\item Pala dixit
\label{sec:org0bb5c09}
stessa cosa
\end{enumerate}

\subsubsection{Terza query}
\label{sec:org8250404}
il medico che ha visitato il maggior numero di diversi pazienti

a sto punto questo schema lo facciamo un design pattern quante volte lo rifila.
\begin{verbatim}
create view medicoPazienti(medico, numero) as
select medico, count(distinct paziente)
from visita
group by medico

select medico
from medicoPazienti
where numero = (select max(numero) form medicoPazienti)
\end{verbatim}

\begin{enumerate}
\item Pala dixit
\label{sec:org554db4f}
si fa una vista in cui si contano per i medici il numero di pazienti (diversi) visitati e
si fa il max di questa vista

\begin{verbatim}
create view numPazienti as
select medico, count(distinct paziente) num
from visita
group by medico

select medico
from numPazienti
where num = (select max(num) form numPazienti)
\end{verbatim}

se vuoi contare tutti i medici, magari anche quelli che non hanno fatto nessuna visita,
allora fai un join esterno.

nell'economia della query in questione non ha molto senso vedere anche i medici che non
hanno mai fatto nessuna visita, quindi il join esterno possiamo evitarlo.
\end{enumerate}

\subsubsection{Quarta query}
\label{sec:orgaf01ec0}
il paziente che è stato visitato dal maggior numero diverso di medici

stessa cosa di sopra ma invertito
\begin{verbatim}
create view numPazienti as
select paziente, count(distinct medico) num -- abbiamo cambiato sta riga
from visita
group by paziente -- sta riga

select paziente -- e sta riga, solo quelle
from numPazienti
where num = (select max(num) form numPazienti)
\end{verbatim}

\subsubsection{Quinta query}
\label{sec:org088b27d}
numero medio di farmaci prescritti al termine delle visite

va fatto un join esterno per contare le visite in cui non prescrivi niente

come al solito creiamo una vista e poi \texttt{select avg(numero)}, siamo al secondo design
patter del Pala.
\begin{verbatim}
create view numFarmaci(visita, numero) as
select v.codice, count(farmaco)
from visita v left join prescrizione f on v.codice = f.visita
group by v.codice

select avg(numero) as HaiCapitoVaBeneComeAlSolito
from numFarmaci
\end{verbatim}

Il Pala ha fatto la stessa cosa

\subsubsection{Sesta query}
\label{sec:org47a1901}
Uguale alla quinta. Il numero medio di controlli fatti in una visita

\subsection{Prova compitino X}
\label{sec:org768b289}
\begin{itemize}
\item Wine : \uline{code}, name, color, year
\item Color : \uline{code}, name
\item Grape\textsubscript{variety} : \uline{code}, name
\item Wine\textsubscript{grape}\textsubscript{variety} : \uline{wine}, \uline{grape\textsubscript{variety}}
\item Food : \uline{code}, name, recipe, food, category
\item Food\textsubscript{category} : \uline{code}, name
\item Wine\textsubscript{food}\textsubscript{pairing} : \uline{code}, wine, food, rating
\item Rating : \uline{code}, description
\end{itemize}

io posso avere un rating a una stessa coppia più volte, posso dire più volte che le
lasagne con l'Ambrusco ci stanno da 10.

\subsubsection{Prima}
\label{sec:orgf96fe0e}
Codici dei fini e delle pietanze che abbinati hanno avuto il massimo numero di valutazioni
'Perfect'

\begin{verbatim}
create view perfects(wine, food, numero)
select wine, food, count(rating)
from wine_food_pairing
where rating = 'Perfect'
-- qui ho fatto una piccolissima minchiata
-- il codice non dice un cazzo su che rating ha l'abbinamento
-- quello devi vederlo dalla tabella rating

select wine, food, numero
from perfects
where numero = (select max(numero) from perfects)
\end{verbatim}

\begin{enumerate}
\item Pala dixit
\label{sec:orgd3b0148}
\begin{verbatim}
create view pairings as
select wine, food, count(r.code) number -- funziona anche count(*), non count distinct (?)
from wine_food_pairing wfp, rating r
where rating = r.code and description = 'Perfect'
group by wine,food

select wine,food
from pairings
where number = (select max(number) from pairing)
\end{verbatim}
\end{enumerate}



\subsubsection{Seconda}
\label{sec:org76bf91e}
il numero di vini prodotti usando un solo vitigno (\texttt{grape\_variety})

\begin{verbatim}
select wine from wine_grape_variety having count(grape_variety) = 1
\end{verbatim}

si poteva usando con \texttt{grape\_variety} due volte
\begin{enumerate}
\item Pala dixit
\label{sec:org22b7e09}
hey indovina un po' mi sono scordato il group by

\begin{verbatim}
select wine from wine_grape_variety having count(grape_variety) = 1 group by wine
\end{verbatim}

altrimenti

\begin{verbatim}
create view vitigni_per_vino as 
select wine, count(*) as numero
from wine_grape_variety
group by wine

select count(wine)
from vitigni_per_vino
where numero = 1
\end{verbatim}

per fare la conta devi fare una vista, o un (se sei scemo)
\begin{verbatim}
select count(*)
from (select wine from wine_grape_variety having count(grape_variety) = 1 group by wine) qualcosa
\end{verbatim}
\end{enumerate}

\subsubsection{Terza}
\label{sec:orgd7f0c7e}
per ogni vino, il codice, il nome, ed il numero di valutazioni di abbinamenti che
coinvolgono quel vino, mostrando 0 per i vini che non compaiono negli abbinamenti

\begin{verbatim}
select w.code, w.name, count(wfp.rating)
from wine w left join wine_food_pairing wfp on (wfp.wine = w.codice)
group by w.code, w.name
\end{verbatim}

\begin{enumerate}
\item Pala dixit
\label{sec:org07b3461}
viene esplicitato per gli 0

va usato wine perchè chiede codice e nome, e visto che chiede tutti i vini va obbligata
ogni riga della tabella wine con un join esterno

e contiamo \texttt{wine\_food\_pairing.code} per contare gli 0 , non può essere sostituito con
\texttt{count(*)} visto che va considerato il count 0 per i vini non cosati

\begin{verbatim}
select w.code, w.name, count(wfp.code)
from wine w left join wine_food_pairing wfp on wine = w.code
group by w.code, w.name
\end{verbatim}
\end{enumerate}
\end{document}