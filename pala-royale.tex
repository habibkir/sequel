% Created 2022-11-09 Wed 15:08
% Intended LaTeX compiler: pdflatex
\documentclass[11pt]{article}
\usepackage[utf8]{inputenc}
\usepackage[T1]{fontenc}
\usepackage{graphicx}
\usepackage{longtable}
\usepackage{wrapfig}
\usepackage{rotating}
\usepackage[normalem]{ulem}
\usepackage{amsmath}
\usepackage{amssymb}
\usepackage{capt-of}
\usepackage{hyperref}
\author{Biggus Diccus}
\date{\today}
\title{}
\hypersetup{
 pdfauthor={Biggus Diccus},
 pdftitle={},
 pdfkeywords={},
 pdfsubject={},
 pdfcreator={Emacs 28.1 (Org mode 9.5.5)}, 
 pdflang={English}}
\begin{document}

\tableofcontents

\section{vino}
\label{sec:org334d11d}
\subsection{Note}
\label{sec:org755696c}
nella prima basta fare un abbinamento per \ldots{}
e andare a fare count \textbf{distinct} di vino = 1
ricorda, quando hai un conteggio
\begin{itemize}
\item count
\item count *
\item count distinct
\end{itemize}

\subsection{Prima}
\label{sec:org7703864}
il codice delle pietanze abbinate a un solo vino
in questo caso è stata abbinata dieci volte allo stesso vino, quel vino è sempre lo
stesso, non bisogna fare i distintoni con questi vini, in questo caso è questo caso per
questo caso, del caso.

\begin{verbatim}
select food
from wine_food_pairing
group by food
having count(distinct wine) = 1

\end{verbatim}

\subsection{Seconda}
\label{sec:org60f75c6}
codice del fino che più spesso compare nell evautazoini degli abbinamenti

\begin{verbatim}
create view volte_che_compare(vino, numero) as
select w.code, count(wfp.code)
from wine w left join wine_food_pairing wfp on w.code = wfp.wine
group by w.code;
\end{verbatim}

trovi il massimo con

\begin{verbatim}
select vino
from volte_che_compare
where numero = (select max(numero) from volte_che_compare)
\end{verbatim}

\subsubsection{Pala dixit}
\label{sec:org5bb020b}
conto codice, siccome codice è una chiave primaria
\begin{itemize}
\item \texttt{count(code)}
\item \texttt{count(distinct code)}
\item \texttt{count(*)}
\end{itemize}
non cambia   

\begin{verbatim}
create view winePairings as
select wine, count(*) number
from wine_food_pairing
group by wine
\end{verbatim}

\begin{verbatim}
select wine
from winePairings
where number = (select max(number) from winePairings);
\end{verbatim}

mi raccomando, è la select che ritorna lo scalare, devi confrontare il valore con il
valore scalare ritornato dalla select, quindi non fare
\begin{verbatim}
where numero = max(numero)
\end{verbatim}
perchè max ha senso solo come "subroutine" di una select, i valori vengono dalla select,
tutto viene dalla select, tutto arriva alla select, select.

ricorda che nella target list puoi fare operazoini aritmetiche anche di query annidata

\subsection{Terza}
\label{sec:org4c6b87e}

per ogni possibile giudizio, il codice e la descrizione ed il numero di valutazoini di
abbinamenti con quel giudizio, mostrando 0 per i giudizi mai assegnati

\begin{itemize}
\item rating legato con vincolo di integrità referenziale, rating indica tutti i possibili
tipi di descrizione
\end{itemize}

\begin{verbatim}
select r.code, r.description, count(wfp.code)
from ratint r left join wine_food_pairing wfp on wfp.rating = r.code
group by r.code, r.description;
\end{verbatim}

\subsubsection{Pala dixit}
\label{sec:orgfd91377}
si chiede esplicitamente che compaiano tutti i possibili giudizi, il che vuol dire join
esterno, che può essere sinistro o destro a seconda dell'ordine degli addendi, il
risultato \emph{cambia}.

il numero di \ldots{} con quel giudizio, il raggruppamento va fatto per rating.code,
rating.giudizio, qual'è la variabile da contare? wine\textsubscript{food}\textsubscript{pairing.code}, perchè altrimenti
che ho fatto l'esetrno a fare

distinct o non distinct non serve a niente, qui \texttt{code} è univoco, quando non è nullo, ma
può essere nullo per i rating che non sono mai usati, quindi non si fa count(*)

mi raccomando in questi casi qua, lo ha già detto 10\textsuperscript{4} volte ma nei join esterni che qui
ha \texttt{rating = r.code}, attenzione che se faccio il group by rating

\begin{quote}
se uno per una query del genere mi fa 3 viste,
un punto glielo tolgo anche per soddisfaizione personale
 -- Nietzsche, così parlò Zarathustra
\end{quote}

\section{ISBN}
\label{sec:orgf79e129}
estrarre il numero di autocitazioni
un autorie che cita in un articolo un altro articolo di se stesso

\begin{itemize}
\item Editor, codice, nome, indirizzo, paese
\item Journal, isbn, titolo, editor
\item JournalIssue (un uscita al mese) , journal, volume, numero, anno
\item JournalPaper, chiave journal, volume, numero, id, titolo
\item Autore, nome, cognomme, affiliazione, paese
\item ArticoloAutore, journal, volume, numero, id, autore
\item Citazione, journal, volume, numero, id (citatore),
journalCitato, numeroCitato, idCitato (citato)
\end{itemize}

può darsi che nei primi quattro codici e nei secondi quattro codici trovi lo stesso autore

\subsection{Prima}
\label{sec:org13f04a9}
Estrarre il numero di autocitazioni dell'autore PLAPTR001

\begin{itemize}
\item serve la tabella citazioni
\begin{itemize}
\item serve trovare l'autore del citato, usando tabella articoloAutore
\end{itemize}
\end{itemize}

un abominio naive è
\#+begin\textsubscript{src} sql
      select count(*)
    from citazione c, articoloAutore a1, articoloAutore a2
    where
  a1.journal = c.journal
  a1.volume = c.volume
  a1.numero = c.numero
  a1.id = c.id
  and
  a2.journal = c.journalCitato 
  a2.nuemro = c.nuemroCitato 
  a2.id = c.idCitato 

\subsubsection{Pala dixit}
\label{sec:org12999ce}
tabella citazione combinata a una prima istanza di paperAuthor

a destra lo estendo, utilizzando una seconda istanza di paperAuthor, con l'autore del
papaer citato

(quando parla del codice del paper parla di 4 codici)
in modo da avere su una stessa riga una priam e una seconda istanza di paperautor, il
codice del paper di cui è autore, il codice del paper citato, e l'autore del paper citato,
se impongo che gli autori siano uguali e uguali all'autore in questione.
poi mi resteranno le righe di lui che cita se stesso

\begin{verbatim}
select count(*) from
PaperAuthro pa1,
Citation c1,
PaperAuthro pa2
where
-- i primi quattro cosi di citazione indicano il primo articolo
-- della relazione primo -> cita -> secondo
and
-- gli altri quattro cosi di citazoine indicano il secondo articolo
-- della relazione primo -> cita -> secondo
and pa1.author = pa2.author
and pa1.author = 'PLAPTR001';
\end{verbatim}

espansa con la chiarezza di una macro del c++ abbiamo
\begin{verbatim}
select count(*) from
PaperAuthro pa1,
Citation c1,
PaperAuthro pa2
where
pa1.journal = c1.journal and
pa1.volume = c1.volume and
pa1. number = c1.number and
pa1.id = c1.id
and
pa2.journal = c1.citedJ and
pa2.volume = c1.citedV and
pa2. number = c1.citedN and
pa2.id = c1.citedID
and
pa1.author = pa2.author and pa1.author = 'PLAPTR001';
\end{verbatim}

con l'espansione cubica di prodotti cartesiani qui comunque mi sa abbiamo fatto un
DDOS. Bello il triplo join, implicito tra l'altro.

il natural join funzionava per il primo join ma non per il secondo join
il natural join permette di non specificare la condizione di join che è implicitamente
data dall'equivalenza degli attributi con lo stesso nome

\subsection{Seconda}
\label{sec:org8c3a6d2}
Estrarre il numero medio di autori di dei papers pubblicati negli issues del 2016 del
jounrnal con ISBN '001-001-001'

possiamo fare una vista con
\[
paper\ del\ 2016 \to #(autori)
\]

questa la famo con

\subsubsection{Pala dixit}
\label{sec:org273b69e}
\begin{itemize}
\item va utilizzato journalIssue perchè l'anno compare lì
\item va usato paperAuthor perchè l'autore compare lì
\end{itemize}

tutti i dati di journalPaper sono messi implicitamente in stocazzo

\begin{verbatim}
create view author_per_paper(journal, volume, number, id, numAuth) as
select journal, volume, number, id, count(author)
from paperAuthor
group by
journal, volume, number, id
\end{verbatim}

poi boh

\subsection{Terza}
\label{sec:org1f80382}
Estrarre id e issue del paper citato il minor numero di volte
\begin{verbatim}
create view cite_per_paper(journal, volume, number, id, cit_numero) as
select citedJ, citedV, citedN, citedID, count(journal)
from citation
group by citedJ, citedV, citedN, citedID
\end{verbatim}

\begin{verbatim}
select journal, volume, number, id, cit_numero
from cite_per_paper
where cit_numero = (select min(cit_numero) from cite_per_paper);
\end{verbatim}

cosa cazzo è appena diventato questo corso, stavamo così allegri a fare i select where
numero = min\ldots{} e adesso boh.

\subsection{Ultima select}
\label{sec:org351b61a}
estrarre il numero di citazoini ai paper dell'autore col solito id

pseudocodice
\begin{verbatim}
select count(*)
from citazoini, paperAuthor
where citazioni.citato = paperAuthor.paper
and paperAuthor.paper = 'quello'
\end{verbatim}

codice
\begin{verbatim}
select count(*)
from citazoini, paperAuthor
where
citazioni.citedJ = paperAuthor.journal
citazioni.citedV = paperAuthor.volume
citazioni.citedN = paperAuthor.number
citazioni.citedID = paperAuthor.id
and paperAuthor.paper = 'PLAPTR001' -- Pala Pietro number 1
\end{verbatim}

\subsubsection{Pala dixit}
\label{sec:orgc628b82}
sono stupido, gli \texttt{and}
\begin{verbatim}
select count(*)
from citazoini, paperAuthor
where
citazioni.citedJ  = paperAuthor.journal and
citazioni.citedV  = paperAuthor.volume and
citazioni.citedN  = paperAuthor.number and
citazioni.citedID = paperAuthor.id and
paperAuthor.author = 'PLAPTR001'
\end{verbatim}

(lui lo ha fatto con gli alias, ma sticazzi non riscrivo la query)

\section{Per i compitini}
\label{sec:orgea3a55c}
troverete dei fogli sul banco
sedetevi solo dove trovi fogli

a quel punto lui proietta la descrizoine della base dati
si legge la descrizione
lui fa commenti

poi tutto bellino, quando nessuno ha più domande si da il via allo svolgimenti del corso,
e lì giri il foglio

prima di tutto scrivi nome, cognome, e matricola dello studente sul foglio in modo da
poterlo individuare univocamente all'interno della tabella compiti.

chi volesse arrivare prima sappia che appena si siede sta lì, la scelta di posaculo è
vincolante.

l'idea della durata è mezz'ora

se poi si dilunga sull'ora si dilunga sull'ora

poi se la sai fare ci vorrebbero anche 3-5 minuti (minchia)

se volete potete portarvi fogli bianchi, suggerisce di portarli

alla scdenza del compito arrivate tutti alla cattedra e consegnate, post fine hai 2-3
minuti poi se non consegni so' cazzi tua.
\end{document}